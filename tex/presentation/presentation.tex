\documentclass{beamer}

% This file is a solution template for:

% - Talk at a conference/colloquium.
% - Talk length is about 20min.
% - Style is ornate.


\mode<presentation>
{
  \usetheme{Berkeley}
  % or ... Warsaw, Antibes, Bergen, Berkeley, Berlin

  \setbeamercovered{transparent}
  % or whatever (possibly just delete it)
}


\usepackage[english]{babel}
\usepackage[latin1]{inputenc}
\usepackage{color}

\usepackage{times}
\usepackage[T1]{fontenc}
% Or whatever. Note that the encoding and the font should match. If T1
% does not look nice, try deleting the line with the fontenc.

\usepackage{graphics}

\title[i]{A Numerical Study of the Accuracy of Divergence-Free Kernel 
Approximations}
\subtitle{}

\author[Arthur Mitrano]
{Arthur Mitrano}
% - Give the names in the same order as the appear in the paper.
% - Use the \inst{?} command only if the authors have different
%   affiliation.

\institute[Arizona State University] % (optional, but mostly needed)
{
  School of Mathematical and Statistical Sciences\\
  Arizona State University}
% - Use the \inst command only if there are several affiliations.
% - Keep it simple, no one is interested in your street address.

\date[AN13] % (optional, should be abbreviation of conference name)
{April, 2013}
% - Either use conference name or its abbreviation.
% - Not really informative to the audience, more for people (including
%   yourself) who are reading the slides online

\subject{}
% This is only inserted into the PDF information catalog. Can be left
% out. 



% If you have a file called "university-logo-filename.xxx", where xxx
% is a graphic format that can be processed by latex or pdflatex,
% resp., then you can add a logo as follows:

% \pgfdeclareimage[height=0.5cm]{university-logo}{university-logo-filename}
% \logo{\pgfuseimage{university-logo}}



% Delete this, if you do not want the table of contents to pop up at
% the beginning of each subsection:

% If you wish to uncover everything in a step-wise fashion, uncomment
% the following command: 

%\beamerdefaultoverlayspecification{<+->}


\begin{document}

\begin{frame}
  \titlepage
\end{frame}

\begin{frame}{Outline}
  \tableofcontents
  % You might wish to add the option [pausesections]
\end{frame}


% Structuring a talk is a difficult task and the following structure
% may not be suitable. Here are some rules that apply for this
% solution: 

% - Exactly two or three sections (other than the summary).
% - At *most* three subsections per section.
% - Talk about 30s to 2min per frame. So there should be between about
%   15 and 30 frames, all told.

% - A conference audience is likely to know very little of what you
%   are going to talk about. So *simplify*!
% - In a 20min talk, getting the main ideas across is hard
%   enough. Leave out details, even if it means being less precise than
%   you think necessary.
% - If you omit details that are vital to the proof/implementation,
%   just say so once. Everybody will be happy with that.

\section{Introduction}
\begin{frame}{Why use divergence-free schemes}
  \begin{itemize}
    \item Assure that the divergence of a vector field is zero.

    \item Useful in simulation of incompressible fluid flows.

    \item Enforcement of zero divergence is necessary for numerical stability.

    \item Divergence-free interpolants based on polynomials and radial basis
      functions.
  \end{itemize}	
\end{frame}
% The idea of this slide is give a brief reason of why is important to pay
% attention to divergence-free methods. It might be better to include some
% examples on the slides to not have to memorize everything.

\begin{frame}{Test problem}
  \begin{itemize}  
    \item Divergence-free vector field:
      \begin{equation*}
	\mathbf{f}(x,y) = \left( \frac{\sin(k_1 x)\cos(k_2 y)}{k_1}, 
	-\frac{\cos(k_1 x)\sin(k_2 y)}{k_2}\right)
      \end{equation*}
  \end{itemize}

  \begin{center}
    \includegraphics[width=.55\linewidth]{../img/vectorFieldAndInterpolant}
  \end{center}
\end{frame}
% In this slide we explain what we are doing numerically, that is, explain which
% test problem we are using and what we are investigating:
% 1) Looking for a RBF or a polynomial divergence-free interpolants
% 2) Calculation the numerical precision of the derivatives of those
%    interpolants.
% Maybe is better to put a extra slide saying those two topics instead just talk
% about them.

\section{RBFs interpolants}
\subsection{Basic of RBF interpolation}
\begin{frame}{Basic of RBF interpolation}
  \begin{equation*}
    s_{f,N}(x) = \sum_{k=1}^{N} c_{k}\varphi(\varepsilon ||x - x_{k}||)
  \end{equation*}

  \begin{itemize}
    \item Coefficients $c_k$ are calculated such that $s_{f,N}(x_k) = f(x_k)$,
      for $k = 1,\ldots,N$.

    \item $\varepsilon$ is the shape parameter.

    \item Examples of basic functions:
      \begin{align*}
        \varphi(r) &= \exp(-r^{2}) &\text{Gaussians} \\
        \varphi(r) &= 1/(1 + r^{2}) &\text{inverse quadrics} \\
        \varphi(r) &= \sqrt{1 + r^{2}} &\text{multiquadrics}
      \end{align*}
  \end{itemize}
\end{frame}
% Here is important to talk about the linear system that we will fall, how the
% shape parameter $\varepsilon$ changes the interpolant and mention some
% of basic functions. Note that we going to talk more about those on the coming
% slides, so no need to get in details here.

\begin{frame}{Some properties}
  \begin{itemize}
    \item Enforcing the interpolation conditions lead to the linear system 
      $A\mathbf{c} = \mathbf{f}$.
      
    \item The interpolation matrix $A$ is positive-definite for some choice of
      basic function $\varphi$.

    \item $\varepsilon$ has a key role in the accuracy of the approximation and
      the conditioning of $A$.
  \end{itemize}
\end{frame}
% Here we must emphasize that the interpolation matrix is invertible for
% positive-definite functions $\varphi$, however, the depending of the shape
% parameter chosen the condition number of $A$ can be extremely high.

\begin{frame}{Effect of the shape parameter for $\varepsilon$}
  \begin{center}
    \includegraphics[width=\linewidth]{../img/understandingShapeParameter}\\
    $f(x) = e^x + 0.1\sin(10x)$.
  \end{center}

  \begin{itemize}
    \item Large value of $\varepsilon$ leads to a localized interpolant, and a
      small value gives a global approximation.
  \end{itemize}
\end{frame}
% Here we illustrate the effect of the shape parameter $\varepsiolon$. It is
% important to remark that $\varepsilon \rightarrow 0$ the RBF becomes flatter
% giving a global approximation, and if $\varepsilon \rightarrow \infty$ the RBF
% becomes a spike and the interpolant becomes localized on the data sites.

\subsection{Divergence-free RBF interpolant}
\begin{frame}{Divergence-free RBF interpolant}
  \begin{itemize}
    \item $\mathbf{t}_1,\ldots,\mathbf{t}_N$ are samples of a vector field at
      the points $\mathbf{x}_1,\ldots,\mathbf{x}_N$.
      
    \item The divergence-free interpolant is given by
      \begin{equation*}
	\mathbf{t}(\mathbf{x}) = \sum_{k = 1}^{N}
	{\Psi(\mathbf{x},\mathbf{x}_k)\mathbf{s}_k}
      \end{equation*}
      where $\Psi(\mathbf{x},\mathbf{y}) = X_\mathbf{n_x}
      (-\nabla\nabla^T\psi(\mathbf{x} - \mathbf{y}))X_{\mathbf{n_y}}^T$ with
      $X_\mathbf{a}\mathbf{b} = \mathbf{a}\times\mathbf{b}$. \emph{Narcowich,
      Ward and Wright (2007), ``Divergence-free RBFs on surfaces''}.
  \end{itemize}
\end{frame}
% The objective here is to introduce the divergence-free RBF interpolant
% mentioning that it works for surfaces on $\mathbb{R}^3$ and that $\varphi$ is
% chosen to be a positive-definite RBF, which is going to lead to non-singular
% system of linear equations. Note also that the columns of the kernel are
% divergence-free.

\begin{frame}{Kernel for the 2D case}
  \begin{equation*}
    \Psi(\mathbf{x},\mathbf{y}) = -F(r)I - G(r)(\mathbf{x} - 
    \mathbf{y})(\mathbf{x} - \mathbf{y})^T
  \end{equation*}

  \begin{equation*}
    F(r) = \frac{1}{r}\phi'(r), \quad G(r) = 
    \frac{1}{r}\left(\frac{1}{r}\phi'(r)\right)' = \frac{1}{r}F'(r)
  \end{equation*}
\end{frame}
% This slide is just to show the kernel for two dimensions, maybe is better to
% put together with the previous slide.

\begin{frame}{New linear system}
  \begin{itemize}
    \item To calculate the $\mathbf{s}_k$ vectors, we evaluate the interpolant
      on the nodes $\mathbf{x}_j$.
      \begin{gather*}
        \mathbf{t_j} = (u_j,v_j), \quad \mathbf{s}_k = (s_k^u,s_k^v)\\
        \mathbf{d} = A\mathbf{c},\\
        \mathbf{d} = [u_1 \ v_1 \cdots u_N \ v_N]^{T}, \quad
        \mathbf{c} = [s_1^u \ s_1^v \cdots s_N^u \ s_N^v]^{T}
      \end{gather*}

    \item $A$ is a $2N\times 2N$ matrix composed of blocks $\Psi(\mathbf{x}_j,
      \mathbf{x}_k)$.

    \item \emph{Narcowich, Ward and Wright (2007)} prove that $A$ is positive-definite
      if $\varphi$ is positive-definite.
    \end{itemize}
\end{frame}
% From the interpolation conditions we can extract (after some organization) a
% linear system of size $2N\times 2N$. We construct the interpolation matrix
% using the blocks generated by the kernel when evaluate at the data sites.
% To follow the structure of $A$, we rewrite the vectors of $\mathbf{t}_j$ and
% $\mathbf{s}_k$ as shown on the equations.

\begin{frame}{Differentiation}
  \begin{itemize}
    \item Taking the derivative of the divergence-free interpolant and
      evaluating at the data sites:
      \begin{gather*}
	\mathbf{d}_x = A_x\mathbf{c} \quad \rightarrow \quad \mathbf{d}_x =
	A_x A^{-1}\mathbf{d} \\
	\mathbf{d}_x = [u_x(\mathbf{x}_1) \ v_x(\mathbf{x}_1) \cdots 
	u_x(\mathbf{x}_N) \ v_x(\mathbf{x}_N)]^T
      \end{gather*}

    \item $A_x$ is $2N\times 2N$ matrix with $2x2$ block entries:
      \begin{equation*}
        (A_x)_{ik} = \frac{\partial \Psi}{\partial x}
	(\mathbf{x}_i,\mathbf{x}_k), \quad i,k \in \left\{1,\ldots,N\right\}.
      \end{equation*}
  \end{itemize}
\end{frame}
% On this slide we need to mention that when we want to differentiate the RBF
% interpolant (divergence-free or not) we can reuse the coefficients already
% calculated, i.e., we do not need to invert another matrix.

\begin{frame}{Error decay of divergence-free RBF interpolant derivatives}
  \begin{center}
    %\includegraphics[width=0.55\linewidth]{../img/decayRBFderivatives}
    $\varepsilon = 2$
  \end{center}
\end{frame}
% This plot is to show de decay of the derivatives of the first component of the
% divergence-free RBF interpolant. What we should remark here is that we have
% spectral convergence, till we get to a plateau due to the ill conditioning of
% the interpolation matrix. 
% I MIGHT CHANGE THIS SLIDE TO THE ERROR DECAY OF INTERPOLATION, INSTEAD OF
% DERIVATIVES. THE REASON FOR THAT IS IF THE INTERPOLATION MATRIX HAS A LARGE
% CONDITION NUMBER, THERE IS NO REASON TO HOPE THE DIFFERENTIATION MATRIX
% (BASED ON A) WILL PERFORM WELL.

\subsection{Condition number of interpolation matrix}
\begin{frame}{Condition number of A}%{Analizing the spectral coefficients}
  \begin{table}[hptb]
    \begin{center}
      \begin{tabular}{l||l|l}
        \multicolumn{1}{c||}{$\sqrt{N}$} & 
	\multicolumn{1}{c|}{$Cond(A_{divFree})$} & 
	\multicolumn{1}{c}{$Cond(A_{trad})$} \\
        \hline \hline
        %$2$ & $1.35\times 10^{0}$ & $1.08\times 10^{0}$ \\
        $3$ & $3.82\times 10^{1}$ & $9.79\times 10^{0}$ \\
        $4$ & $7.36\times 10^{3}$ & $4.41\times 10^{2}$ \\
        $5$ & $4.49\times 10^{6}$ & $4.94\times 10^{4}$ \\
        $6$ & $4.87\times 10^{9}$ & $1.06\times 10^{7}$ \\
        $7$ & $8.11\times 10^{12}$ & $3.72\times 10^{9}$ \\
        $8$ & $1.76\times 10^{16}$ & $1.91\times 10^{12}$ \\
        $9$ & $1.93\times 10^{17}$ & $1.40\times 10^{15}$ \\
        $10$ & $2.52\times 10^{18}$ & $1.23\times 10^{17}$ \\
        $11$ & $2.33\times 10^{19}$ & $5.19\times 10^{17}$ \\
        $12$ & $5.04\times 10^{18}$ & $1.81\times 10^{19}$ \\
        $13$ & $1.94\times 10^{20}$ & $5.43\times 10^{18}$ \\
        %$14$ & $4.72\times 10^{19}$ & $8.88\times 10^{18}$ \\
        %$15$ & $2.63\times 10^{19}$ & $5.45\times 10^{19}$   
    \end{tabular}   
    \end{center}
    \caption{Condition number of the interpolation matrices of the 
    divergence-free and traditional methods. $\varepsilon = 2$}
  \end{table}
\end{frame}
% The table show how the condition number of the interpolation matrix for the
% divergence-free ($A_{divFree}$) and the traditional ($A_{trad}$) cases. We see
% that the condition number of $A_{divFree}$ grows a little bit more faster than
% $A_{trad}$. The idea is to pass that we need to deal with the ill-conditioning
% issue. One alternative to this is the RBF-FD type methods. Also mention that
% you are using the same shape parameter for all runs, there are better results
% in terms of condition number if we tweak the shape parameter.

\begin{frame}{Condition number of A}
  \begin{table}[hptb]
    \begin{center}
      \begin{tabular}{l||l|l}
        \multicolumn{1}{c||}{$\sqrt{N}$} & 
	\multicolumn{1}{c|}{$Cond(A_{divFree})$} & 
	\multicolumn{1}{c}{$Cond(A_{trad})$} \\
        \hline \hline
        $3$ & $1.1\times 10^{0}$ & $1.0\times 10^{0}$ \\
        $4$ & $3.3\times 10^{0}$ & $1.5\times 10^{0}$ \\
        $5$ & $1.9\times 10^{1}$ & $4.6\times 10^{0}$ \\
        $6$ & $1.5\times 10^{2}$ & $2.2\times 10^{1}$ \\
        $7$ & $2.9\times 10^{3}$ & $1.5\times 10^{2}$ \\
        $8$ & $1.1\times 10^{5}$ & $1.3\times 10^{3}$ \\
        $9$ & $5.8\times 10^{6}$ & $1.6\times 10^{4}$ \\
        $10$ & $4.4\times 10^{8}$ & $2.6\times 10^{5}$ \\
        $11$ & $4.4\times 10^{10}$ & $5.4\times 10^{6}$ \\
        $12$ & $5.7\times 10^{12}$ & $1.4\times 10^{8}$ \\
        $13$ & $9.1\times 10^{14}$ & $4.6\times 10^{9}$ \\
    \end{tabular}
    \end{center}
    \caption{Condition number of the interpolation matrices of the 
    divergence-free and traditional methods. $\varepsilon = 5$}
  \end{table}
\end{frame}
% When we change the shape parameter we can use large number of points before
% the interpolation matrix get ill-conditioned. Note that the condition number
% of the divergence-free method increases faster.

\begin{frame}{Condition number of A}
  \begin{figure}[htb]
    \begin{center}
      \includegraphics[width=0.5\linewidth]{../img/condNumberGrowthE2}
      \includegraphics[width=0.5\linewidth]{../img/condNumberGrowthE5}
    \end{center}
    \caption{Left: $\varepsilon = 2$, right: $\varepsilon=5$.}
  \end{figure}
\end{frame}
% Here we can see clearly that the condition number for the divergence-free
% method increase at a faster rate. To avoid this ill conditioning problem we
% can use the RBF-FD approach

\subsection{RBF-FD}
\begin{frame}{RBF-FD}
  \begin{itemize}
    \item To avoid the ill-condition of the interpolation matrix we might use
      the RBF-FD approach.
  \end{itemize}
  \begin{center}
    \includegraphics[width=0.55\linewidth]{../img/grid}
  \end{center}
\end{frame}
% Here we should explain that the condition number for a few number of points is
% reasonable so it might be a good idea to use a RBF-FD type methods. The
% picture of the grid should help on the discription of the RBF-FD method. The
% gray points represent the whole grid, the ones in blue+red the stencil and the
% one in red the point where we wish to know the derivative.

\begin{frame}{Higher accuracy on some derivatives}
  \begin{center}
   \includegraphics[width=0.85\linewidth]{../img/errorDecayFDRBFdivFree}
  \end{center}
  
  \begin{itemize}
    \item For the $V$ component, the decay is faster on $V_y$.
  \end{itemize}
\end{frame}
% This slide show one of the interesting results of the presentation: we get the
% derivatives related to the divergence-free equation ($u_x+v_y=0$) to converge
% twice as fast then the other ones. This might be due to the extra information
% that we get from the divergence-free condition.


\begin{frame}{Divergence-free information}
  \begin{itemize}
    \item Assume that
      \begin{equation*}
	\left\{
        \begin{aligned}
	  u(x,y) &= u_1(x,y) + u_2(y) \\
	  v(x,y) &= v_1(x,y) + v_2(x)
        \end{aligned}
	\right.
      \end{equation*}

    \item Divergence-free condition: $0 = u_x + v_y = {u_1}_x + {v_1}_y$.

    \item The divergence-free condition do not add information for describing
      $u_2$ and $v_2$, only interpolation does.

    \item We can enlarge the stencil to get the same accuracy of $O(N^{-4})$ 
      for, $u_y$ and $v_x$ since the approximation comes from interpolation
      points only.
  \end{itemize}
\end{frame}
% On this slide we should talk about why we do not get an high accuracy on the
% derivatives not related to the divergence-free equation. Assume there is a
% part of a $u$ depending on y and $v$ depending in x, this way those parts are
% not influenced by the divergence-free condition at all. Therefore, the
% convergence results should match with the ones of tradition interpolation.

\section{Polynomial interpolants}
\begin{frame}{Polynomial interpolants}
  \begin{itemize}
    \item The stream function $\psi$ is can be used to create a divergence-free
      vector field.
      \begin{equation*}
	\psi(x,y) = \sum_{i=0}^N\sum_{j=0}^N a_{ij}x^i y^j.
      \end{equation*}
    
    \item $\mathbf{P} = \nabla \times (0,0,\psi) = (\psi_y,-\psi_x,0)$

    \item The interpolation condition leads to a linear system that might have
      many solutions or no one.

    \item Use the least squares solution.
  \end{itemize}
\end{frame}
% We can always try to generate divergence-free interpolants trying to
% approximate the stream function, on this case we are using a tensor polynomial
% base (I guess we could use other basis, like Chebyshev and Fourier, but I'm
% not sure yet). Since is not an easy task to know if we can find such an
% interpolant, we look for least squares solutions of the linear system in hope
% to at least find the best approximation to it for the degree utilize $N$.

\begin{frame}{Error decay of the derivatives for polynomial interpolant}
  \begin{figure}[htb]
    \begin{center}
      \includegraphics[width=0.5\linewidth]{../img/errorDecayFDdivFree}
      \includegraphics[width=0.5\linewidth]{../img/errorDecayFDdivFreeSmallN}
    \end{center}
    \caption{Left: degree $3$, right: degree $2$.}
  \end{figure}
  
  \begin{itemize}
    \item The degrees of freedom on the stream function let we get better
      results for $u_x$ and $v_y$.
      
    \item If $N$ is not big enough we do not achieve a faster rate of
      convergence for the derivatives.
  \end{itemize}
\end{frame}
% Here we see that using $N$ small or big makes a difference on the rate of
% convergence of the derivatives. For the case of the 9 points stencil, we found
% that with $N=5$ and removing some redundant terms we can get a full rank
% square system $18\times 18$. The numerical results do not differ in terms
% of convergence, nevertheless we don't have a rank-deficient system of 
% $18\times 36$ that we get when we use $N = 5$ without removing the redundant
% terms.

\section{Lebesgue constants}

\begin{frame}{Lebesgue constants}
  \begin{center}
    \includegraphics[width=\linewidth]{../img/lebesgueConstGrowth}
  \end{center}
\end{frame}
% For all cases the Lebesgue constant grows with the amount of points utilized,
% however we see that for different shape parameters we have different rates of
% growth (smaller $\varepsilon$ faster the rate). Moreover, for the polynomial
% interpolant we have largest constants (note that we are using equally-spaced 
% nodes). The results might change for different distribution of nodes.

\subsection{Kosloff \& Tal-Ezer mapping}
\begin{frame}{Using different set of nodes}
  \begin{itemize}
    \item Kosloff \& Tal-Ezer mapping:
      \begin{equation*}
        x_j^{kte(\alpha)} := \frac{\arcsin(\alpha x_j^{cheb})}{\arcsin(\alpha)},
	\quad j = 1,\ldots,N.
      \end{equation*}
      where $x_j^{cheb} = \cos{\frac{\pi(j-1)}{N-1}}$.

    \item As $\alpha \rightarrow 0$ the nodes converge to the Chebyshev points.

    \item If $\alpha = 1$ the nodes are equally-spaced.
  \end{itemize}
\end{frame}
% Talk about the Kosloff & Tal-Ezer mapping saying what happens when you chose
% different values of $\alpha$, in particular the extreme cases of $\alpha = 1$
% (equally-spaced points) and $\alpha \rightarrow 0$ (Chebyshev points).

\begin{frame}{Using different set of nodes}
  \begin{table}[hptb]
    \begin{center}
      \begin{tabular}{l||ll|ll|ll}
        \multicolumn{1}{c||}{$\sqrt{N}$} & 
	\multicolumn{2}{c|}{$\varepsilon = 2$} & 
	\multicolumn{2}{c|}{$\varepsilon = 5$} &
	\multicolumn{2}{c}{$\varepsilon = 10$}\\
        \hline \hline
	$3$ & {\color{blue}$1.7$} & $1.7$ & {\color{blue}$1.0$} & $1.0$ &
	{\color{blue}$1.0$} & $1.0$\\
	$5$ & {\color{blue}$5.8$} & $8.2$ & {\color{blue}$1.6$} & $1.2$ &
	{\color{blue}$1.0$} & $1.0$\\
	$7$ & {\color{blue}$13.6$} & $14.4$ & {\color{blue}$2.9$} & $2.9$ &
	{\color{blue}$1.6$} & $1.0$\\
	$9$ & {\color{blue}$24.0$} & $36.2$ & {\color{blue}$5.4$} & $6.7$ &
	{\color{blue}$1.6$} & $1.2$\\
	$11$ & {\color{blue}$44.4$} & $250.4$ & {\color{blue}$15.7$} & $18.3$ &
	{\color{blue}$1.9$} & $1.9$\\
	$13$ & {\color{blue}$205.8$} & $1.2\times 10^4$ & {\color{blue}$24.7$} &
	$26.3$ & {\color{blue}$2.9$} & $2.9$\\   
      \end{tabular}
    \end{center}
    \caption{{\color{blue}(blue)}: optimized $\alpha$; (black): $\alpha = 1$.}
  \end{table}
\end{frame}
% Here we see that choosing different $\alpha$ gave smaller Lebesgue constants
% for the case $\varepsiolon = 2$ which is close to the polynomial case. For the
% other two values of shape parameter the Lebesgue constant is not high for
% equally-spaced points so the optimization routine to find $\alpha$ doesn't
% need to be used.

\begin{frame}{Using different set of nodes}
  \begin{table}[hptb]
    \begin{center}
      \begin{tabular}{l||ll|l}
        \multicolumn{1}{c||}{$\sqrt{N}$} & 
	\multicolumn{2}{c|}{Polynomial} &
	\multicolumn{1}{c}{$\alpha$}\\
        \hline \hline
	$3$ & {\color{blue}$16.5$} & $16.5$ & $0.382$\\
	$5$ & {\color{blue}$18.0$} & $80.3$ & $0.441$\\
	$7$ & {\color{blue}$25.5$} & $1.4\times 10^3$ & $0.090$\\
	$9$ & {\color{blue}$33.7$} & $1.5\times 10^4$ & $0.600$\\
	$11$ & {\color{blue}$40.2$} & $7.8\times 10^4$ & $0.026$\\
	$13$ & {\color{blue}$45.4$} & $5.5\times 10^5$ & $0.345$\\   
      \end{tabular}
    \end{center}
    \caption{{\color{blue}(blue)}: optimized $\alpha$; (black): $\alpha = 1$.}
  \end{table}
\end{frame}
% As was show on the previous plot, the Lebesgue constant for the polynomial
% case are very large for equally-spaced points. However, after optimize the
% value of $\alpha$, the Lebesgue constants decreased for reasonable values.

\section{Conclusion and future work}
\begin{frame}{Conclusion and future work}
  \begin{itemize}
    \item Conclusion

    \item Future work
  \end{itemize}
\end{frame}
% COMMENTS!!!!

\begin{thebibliography}{1}
  \bibitem{bidiagonalization}  Hnetynkova, I. and Plesinger, M.. 
    \emph{The regularizing effect of the Golub-Kahan iterative bidiagonalization 
      and revealing the noise level in the data.}
      BIT Numer Math (2009) 49: 669-696.
\end{thebibliography}
% COMMENTS ON THE REFERENCES!!!

\end{document}
